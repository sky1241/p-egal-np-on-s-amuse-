\documentclass[11pt,a4paper]{article}
\usepackage[utf8]{inputenc}
\usepackage[T1]{fontenc}
% babel french removed (not installed)
\usepackage{amsmath,amssymb,amsthm}
\usepackage{mathtools}
\usepackage{geometry}
\usepackage{xcolor}
\usepackage{tcolorbox}
\usepackage{enumitem}
\usepackage{fancyhdr}
\usepackage{titlesec}

\geometry{margin=2.2cm, top=2.5cm, bottom=2.5cm}

% Colors
\definecolor{alive}{RGB}{86,180,133}
\definecolor{dead}{RGB}{140,150,180}
\definecolor{both}{RGB}{180,120,200}
\definecolor{accent}{RGB}{70,130,200}
\definecolor{darkbg}{RGB}{245,245,250}

% Section style
\titleformat{\section}{\Large\bfseries\color{accent}}{}{0em}{}[\vspace{-0.5em}\color{accent}\rule{\textwidth}{0.4pt}]
\titleformat{\subsection}{\large\bfseries}{}{0em}{}

% Header
\pagestyle{fancy}
\fancyhf{}
\fancyhead[L]{\small\color{gray} Champ H\'elicoidal Infini --- Mod\`ele Cristal$\leftrightarrow$Mycelium}
\fancyhead[R]{\small\color{gray} Sky --- F\'evrier 2026}
\fancyfoot[C]{\small\color{gray}\thepage}

% Shortcuts
\newcommand{\vF}{\mathbf{F}}
\newcommand{\ex}{\mathbf{e}_x}
\newcommand{\er}{\mathbf{e}_r}
\newcommand{\et}{\mathbf{e}_\theta}
\newcommand{\R}{\mathbb{R}}
\newcommand{\Z}{\mathbb{Z}}
\newcommand{\N}{\mathbb{N}}

\begin{document}

% ============================================================================
% TITLE
% ============================================================================
\begin{center}
{\Huge\bfseries Champ H\'elicoidal Infini}\\[0.6em]
{\Large\color{accent} Mod\`ele Unifi\'e : Cristal $\leftrightarrow$ Mycelium}\\[1.2em]
{\large Sky \quad---\quad 16 f\'evrier 2026}\\[0.3em]
{\small\color{gray} ``Un pas d'homme. Tu avances toujours, mais tout est connect\'e.''}
\end{center}

\vspace{1.5em}

% ============================================================================
\section{G\'eom\'etrie : R\'eseau infini de sph\`eres}
% ============================================================================

Les sph\`eres ne sont pas 26. Elles sont infinies. Chaque sph\`ere est un pas,
identique \`a tous les autres.

\begin{tcolorbox}[colback=darkbg, colframe=accent, title={\textbf{D\'efinitions de base}}]
\textbf{Index :} $\;i \in \Z$ \quad (infini dans les deux directions)

\textbf{Espacement :} $\;\Delta > 0$ \quad (constant, identique partout)

\textbf{Positions :}
\[
  x_i = i\,\Delta, \qquad \mathbf{p}_i = (x_i,\; 0,\; 0)
\]

\textbf{Sph\`ere $i$ :} rayon $R_s > 0$ constant
\[
  S_i = \bigl\{(x,y,z) \in \R^3 \;:\; (x - x_i)^2 + y^2 + z^2 = R_s^2 \bigr\}
\]

\textbf{Axe commun} (la ligne infinie) :
\[
  \ell = \bigl\{(x, 0, 0) \;:\; x \in \R \bigr\}
\]
\end{tcolorbox}

\medskip

\textbf{Propri\'et\'e fondamentale --- Invariance par translation :}
\[
  \boxed{\forall\, n \in \Z, \quad x \;\mapsto\; x + n\Delta \quad\text{laisse le syst\`eme identique.}}
\]
Tu peux \^etre sur n'importe quelle sph\`ere : c'est toujours pareil.

% ============================================================================
\section{Enveloppe continue : le poids local}
% ============================================================================

Chaque sph\`ere irradie autour d'elle. Les zones se chevauchent. Somme infinie
de gaussiennes identiques.

\textbf{Largeur d'influence :} $\lambda > 0$

\textbf{Enveloppe :}
\[
  W(x) = \sum_{i=-\infty}^{+\infty} \exp\!\left(-\frac{(x - i\Delta)^2}{2\lambda^2}\right)
\]

Par p\'eriodicit\'e, cette somme converge et est $\Delta$-p\'eriodique :
\[
  W(x + \Delta) = W(x) \quad \forall\, x \in \R
\]

\textbf{Normalisation :}
\[
  Z = \sum_{k=-\infty}^{+\infty} \exp\!\left(-\frac{(k\Delta)^2}{2\lambda^2}\right), \qquad
  w(x) = \frac{W(x)}{Z}
\]
Ainsi $0 < w(x) \leq 1$, et toutes les sph\`eres ont le m\^eme poids.

% ============================================================================
\section{Coordonn\'ees cylindriques et bases}
% ============================================================================

Autour de l'axe $x$, en tout point $(x, y, z)$ :
\[
  r = \sqrt{y^2 + z^2 + \varepsilon^2} \qquad (\varepsilon > 0 \;\text{\'evite la singularit\'e sur l'axe})
\]

\textbf{Vecteurs de base} (\'ecrits en cart\'esien) :
\[
  \ex = (1,\, 0,\, 0), \qquad
  \er = \Bigl(0,\, \frac{y}{r},\, \frac{z}{r}\Bigr), \qquad
  \et = \Bigl(0,\, -\frac{z}{r},\, \frac{y}{r}\Bigr)
\]

% ============================================================================
\section{Le param\`etre $\alpha$ : mort $\leftrightarrow$ vivant}
% ============================================================================

\begin{tcolorbox}[colback=darkbg, colframe=both, title={\textbf{L'axe fondamental}}]
\[
  \alpha \in [0, 1], \qquad s(\alpha) = 2\alpha - 1 \in [-1, +1]
\]

\begin{center}
\renewcommand{\arraystretch}{1.3}
\begin{tabular}{c|c|c|l}
  $\alpha$ & $s$ & Mode & Interpr\'etation \\
  \hline
  $0$ & $-1$ & \textcolor{dead}{\textbf{Cristal}} & Structure fig\'ee. Stable, morte, rigide. \\
  $\tfrac{1}{2}$ & $0$ & \textcolor{both}{\textbf{Schr\"odinger}} & Les deux \`a la fois. Annulation du flux. \\
  $1$ & $+1$ & \textcolor{alive}{\textbf{Mycelium}} & Flux vivant. Explore, reroute, fragile.
\end{tabular}
\end{center}
\end{tcolorbox}

% ============================================================================
\section{Lois du champ : rayon cible et rotation}
% ============================================================================

\textbf{Rayon cible} (distance d'enroulement autour de l'axe) :
\[
  r^*(x) = r_{\min} + (r_{\max} - r_{\min})\, w(x)
\]

\textbf{Vitesse angulaire locale :}
\[
  \Omega(x) = \Omega_{\min} + (\Omega_{\max} - \Omega_{\min})\, w(x)
\]

% ============================================================================
\section{Variante A --- $\alpha$ global (m\^eme partout)}
% ============================================================================

Un seul param\`etre $\alpha \in [0, 1]$ pour tout le r\'eseau.

\begin{tcolorbox}[colback=darkbg, colframe=dead!70!accent, title={\textbf{Champ vectoriel --- Variante A (global)}}]
\[
  \boxed{
    \vF_A(x,y,z) \;=\; v\,\ex
    \;+\; \kappa\bigl(r^*(x) - r\bigr)\,\er
    \;+\; s\;\Omega(x)\;r\;\et
  }
\]
avec $s = 2\alpha - 1$, \quad $\alpha \in [0,1]$ constant.
\end{tcolorbox}

\textbf{Ligne de flux} (streamline unique sur $\R$) :

Phase accumul\'ee :
\[
  \Theta(x) = \Theta_0 + \frac{s}{v} \int_{0}^{x} \Omega(u)\, du
\]

Courbe h\'elicoïdale infinie :
\[
  \boxed{
    \gamma_A(x) = \Bigl(\, x,\;\; r^*(x)\cos\Theta(x),\;\; r^*(x)\sin\Theta(x) \,\Bigr),
    \quad x \in \R
  }
\]

\textbf{Propri\'et\'es :}
\begin{itemize}[nosep]
  \item $\alpha = 1$ : h\'elice qui tourne sens $+$ (mycelium explore)
  \item $\alpha = 0$ : h\'elice qui tourne sens $-$ (cristal, sym\'etrique)
  \item $\alpha = \tfrac{1}{2}$ : $s = 0$, pas de rotation $\Rightarrow$ le flux avance tout droit
  \item Invariance : $\gamma_A(x + \Delta) = T_\Delta \circ \gamma_A(x)$ \`a une rotation pr\`es
\end{itemize}

% ============================================================================
\section{Variante B --- $\alpha_i$ local (chaque sph\`ere choisit)}
% ============================================================================

Chaque sph\`ere $i \in \Z$ a son propre param\`etre $\alpha_i \in [0, 1]$.
Le champ $\alpha$ est un \emph{champ continu} sur $\R$.

\begin{tcolorbox}[colback=darkbg, colframe=alive!70!accent, title={\textbf{Champ $\alpha$ continu}}]
\[
  \alpha(x) = \frac{
    \displaystyle\sum_{i=-\infty}^{+\infty} \alpha_i \;\exp\!\left(-\frac{(x-x_i)^2}{2\lambda^2}\right)
  }{
    \displaystyle\sum_{i=-\infty}^{+\infty} \exp\!\left(-\frac{(x-x_i)^2}{2\lambda^2}\right)
  }
\]
\end{tcolorbox}

C'est une \textbf{interpolation gaussienne} : entre deux sph\`eres, $\alpha(x)$ transite
contin\^ument de $\alpha_i$ vers $\alpha_{i+1}$.

\textbf{Param\`etre de sens local :}
\[
  s(x) = 2\,\alpha(x) - 1
\]

\begin{tcolorbox}[colback=darkbg, colframe=alive, title={\textbf{Champ vectoriel --- Variante B (local)}}]
\[
  \boxed{
    \vF_B(x,y,z) \;=\; v\,\ex
    \;+\; \kappa\bigl(r^*(x) - r\bigr)\,\er
    \;+\; s(x)\;\Omega(x)\;r\;\et
  }
\]
avec $s(x) = 2\,\alpha(x) - 1$, \quad chaque $\alpha_i \in [0,1]$ ind\'ependant.
\end{tcolorbox}

\textbf{Ligne de flux (variante B) :}

\[
  \Theta_B(x) = \Theta_0 + \frac{1}{v} \int_0^x s(u)\,\Omega(u)\, du
\]

\[
  \boxed{
    \gamma_B(x) = \Bigl(\, x,\;\; r^*(x)\cos\Theta_B(x),\;\; r^*(x)\sin\Theta_B(x) \,\Bigr),
    \quad x \in \R
  }
\]

\textbf{Propri\'et\'es :}
\begin{itemize}[nosep]
  \item Sph\`ere $i$ avec $\alpha_i = 1$ : zone mycelium, le flux tourne et explore
  \item Sph\`ere $j$ avec $\alpha_j = 0$ : zone cristal, structure fig\'ee
  \item Sph\`ere $k$ avec $\alpha_k = \tfrac{1}{2}$ : Schr\"odinger, point de neutralit\'e
  \item La \textbf{transition} entre zones est \textbf{douce} (gaussienne), jamais brusque
  \item L'invariance par translation est \textbf{bris\'ee} : chaque point est unique
\end{itemize}

% ============================================================================
\section{Comparaison des deux variantes}
% ============================================================================

\begin{center}
\renewcommand{\arraystretch}{1.4}
\begin{tabular}{l|c|c}
  & \textbf{Variante A} (global) & \textbf{Variante B} (local) \\
  \hline
  Param\`etre & $\alpha \in [0,1]$ unique & $\{\alpha_i\}_{i\in\Z}$, un par sph\`ere \\
  Sym\'etrie & Invariance par translation & Bris\'ee (chaque sph\`ere diff\`ere) \\
  Champ $s$ & $s = 2\alpha - 1$ (constante) & $s(x) = 2\alpha(x) - 1$ (champ) \\
  Streamline & H\'elice r\'eguli\`ere infinie & H\'elice \`a courbure variable \\
  Analogie & Cristal pur / Mycelium pur & \textbf{Organisme vivant} \\
  M\'etaphore & ``Loi universelle'' & ``Chaque pas d\'ecide'' \\
\end{tabular}
\end{center}

% ============================================================================
\section{Param\`etres \`a fixer}
% ============================================================================

\[
  \underbrace{\Delta,\; R_s}_{\text{g\'eom\'etrie}}, \quad
  \underbrace{\lambda}_{\text{influence}}, \quad
  \underbrace{r_{\min},\; r_{\max}}_{\text{enroulement}}, \quad
  \underbrace{\Omega_{\min},\; \Omega_{\max}}_{\text{rotation}}, \quad
  \underbrace{v}_{\text{flux}}, \quad
  \underbrace{\kappa}_{\text{rappel}}, \quad
  \underbrace{\varepsilon}_{\text{r\'egul.}}, \quad
  \underbrace{\alpha \;\text{ou}\; \{\alpha_i\}}_{\text{mort/vivant}}
\]

\medskip

\begin{tcolorbox}[colback=darkbg, colframe=gray, title={\textbf{Cas simplifi\'e : ``tout \'egal, tout infini''}}]
Si tu veux que la structure soit strictement identique partout :
\[
  r_{\min} = r_{\max} = r_0, \qquad \Omega_{\min} = 0, \quad \Omega_{\max} = \Omega_0
\]
Le rayon est constant $r^*(x) = r_0$. Seule la rotation est modul\'ee par $w(x)$.

L'h\'elice devient :
\[
  \gamma(x) = \Bigl(x,\;\; r_0 \cos\Theta(x),\;\; r_0 \sin\Theta(x)\Bigr)
\]
avec $\Theta(x) = \Theta_0 + \frac{s}{v}\int_0^x \Omega_0\, w(u)\, du$.

\medskip
\textbf{Encore plus simple :} $w(x) = 1$ partout (influence uniforme) donne une h\'elice parfaite de pas constant $p = 2\pi v / (s\,\Omega_0)$.
\end{tcolorbox}

% ============================================================================
\section{Note : pourquoi mycelium et pas cristal}
% ============================================================================

Le cristal r\'ep\`ete. Le mycelium \emph{connecte}.

Le cristal est $W(x+\Delta) = W(x)$. Le mycelium est $\alpha_i \neq \alpha_j$ ---
chaque point du r\'eseau d\'ecide s'il explore ou s'il se fige. La variante B
est un organisme : des zones cristallines (os, structure) coexistent avec des
zones vivantes (flux, croissance). Les deux sont n\'ecessaires.

\vspace{0.5em}

Le param\`etre $\alpha$ n'est pas un choix binaire. C'est un \textbf{gradient continu}
entre la mort et la vie. Exactement comme Schr\"odinger : avant la mesure, c'est
les deux. Apr\`es, c'est l'un ou l'autre. Mais le syst\`eme entier ---
le r\'eseau infini --- est \textbf{toujours} les deux \`a la fois.

\[
  \boxed{
    \text{Mort} \;\xleftrightarrow[\quad\alpha \in [0,1]\quad]{}\; \text{Vivant}
  }
\]

\vfill
\begin{center}
\small\color{gray}
``Faut imaginer que c'est infini.''
\end{center}

\end{document}
